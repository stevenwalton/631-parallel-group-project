\documentclass[10pt,twocolumn,letterpaper]{article}

\usepackage{times}
\usepackage{epsfig}
\usepackage{graphicx}
\usepackage{amsmath}
\usepackage{amssymb}
\graphicspath{{Images/}{../Images/}}
\usepackage{subcaption} % for subfigure
\usepackage{stackengine} 
\usepackage{listings}

% Include other packages here, before hyperref.

% If you comment hyperref and then uncomment it, you should delete
% egpaper.aux before re-running latex.  (Or just hit 'q' on the first latex
% run, let it finish, and you should be clear).
\usepackage[breaklinks=true,bookmarks=false]{hyperref}

%\def\httilde{\mbox{\tt\raisebox{-.5ex}{\symbol{126}}}}

% Pages are numbered in submission mode, and unnumbered in camera-ready
%\ifcvprfinal\pagestyle{empty}\fi

%%%%%%%%% TITLE
\title{Parallelizing Neural Networks with OpenMP and CUDA\\
\small{A Comparison Of Implementations and Optimizations}}

\author{Luis Guzman\\
University of Oregon\\
{\tt\small lguzman@uoregon.edu}
\and
Steven Walton\\
University of Oregon\\
{\tt\small swalton2@uoregon.edu}
}

%\thispagestyle{empty}

\begin{document} 
%
\maketitle 
%
\section{Background} 
%
Neural Networks are becoming increasingly popular within computing, with work
affecting almost all sectors of computing. Because of this many researchers have
devoted a lot of time optimizing these algorithms. Deep Neural Networks have
progressed from being academic curiosities to being a cornerstone of many modern
technologies. Because of this researchers are constantly trying to optimize for
different accelerator technologies and platforms. Even long before accelerators
many were trying to parallelize neural networks~\cite{10.1007/BFb0024235}.
While the new resurgence of machine learning partially came from becoming
accelerators like GPUs there are still many innovations for CPUs,
such as Facebook's new speech-to-text (stt) system that works on
CPUs~\cite{fbcpu} which is aimed at home assistants that can't use large and power
hungry accelerators. Algorithms are getting larger and there is continued
efforts to apply machine learning techniques to simpler hardware. Thus for the
foreseeable future optimizing machine learning algorithms on various platforms
will be an active area of research.  

\section{Speeding Up Neural Architectures}
%
There are many areas that neural architectures can be sped up. Many different
calculations can be performed in parallel as they are not dependent on one
another. This can be from the feed forward section to the backwards propagation.
There are also parallel techniques for running the same model on different
machines. There are techniques for data parallelism like batch 
normalizing~\cite{DBLP:journals/corr/IoffeS15} that also improve performance,
which is arguably a speedup too (speedup to a certain convergence value). 

The most computationally intensive part of the neural network is what is
referred to backwards propagation. A neural network simply has a set of priors,
data is fed in, an answer is given, and then you update the priors based on how
wrong the answer is from the true solution. The way to update these priors is
through backwards propagation. Fundamentally the main concept is performed by
using a gradient descent technique, where we use the gradient to take a step in
the direction towards a better solution. Being the most computationally heavy
section of the network, this has become the most heavily optimized. Many have
used optimized libraries like cuBLAS to accelerate training~\cite{5708849} while
others have developed new optimization algorithms~\cite{DBLP:journals/corr/KingmaB14}. 

\section{Proposed Work}
%
Within this work we propose to test and evaluate different levels of
parallelization within a neural network. We will investigate the effects of
different acceleration techniques, including OpenMP and CUDA. For testing we
will compare our network's performance and speed to that of
PyTorch~\cite{paszke2017automatic}, an optimized python library for creating
neural networks. We will test against both PyTorch's CPU and GPU
implementations. To perform an accurate analysis we will compare how different
parts of the network are parallelized or accelerated and thus determine which
sections of the network are benefit the most from parallelization. We will also
profile the code to determine where sections of code need the most speed up.
While we do not expect to match the optimization that PyTorch has, we can
compare naive implementations as well as optimized libraries against
professional software that is commonly used in industry. By comparing both CPU,
serial and parallel, and GPU implementations we can also gain a better
understanding of how neural networks can be optimized for different
architectures. For example, GPUs are great at performing matrix multiplications
while CPUs are more generalized compute units. 

Much of this work will focus on comparing naive implementations vs optimal
solutions, such as the Coppersmith-Winnograd
Algorithm~\cite{COPPERSMITH1990251}. Matrix (and tensor) multiplication is the
backstone of many parts of these neural networks, and comparing naive
implementations to optimized algorithms and optimized software packages gives
insight into how the optimized packages work. By profiling the code we can also
get a better insight into which areas need to be improved the most and we can
quantify how much these types of algorithms affect neural networks. 

The next most important aspect if to parallelize the gradient descent methods.
There has been work in parallelizing Stochastic Gradient Descent
(SGD)~\cite{10.5555/2997046.2997185}, as well as other algorithms like
Adam~\cite{DBLP:journals/corr/KingmaB14}. One simple way to parallelize all of
these algorithms is by performing batching, which we will explore in this work. 

Furthermore, in this work we will do continued research, specifically into the
optimizations and parallelized methods that PyTorch does. At this point we do
not know much about these algorithms but expect that expertise and interest to
grow as the project continues.

\section{Deliverables}
In this work we shall produce source code that performs different optimizations
techniques in a serialized form, with OpenMP, with CUDA, with PyTorch's CPU
implementation, as well as with Pytorch's GPU implementation. 
%
We shall also provide graphs that illustrate these differences for different
network sizes. 
%
For this work we will also summarize our findings as well as further research
and deliver this report by the end of the term.

%
\bibliographystyle{abbrv}
%
\bibliography{references} 
%
\end{document}
